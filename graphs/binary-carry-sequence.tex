\documentclass{standalone}
\usepackage{tikz}
\usetikzlibrary{shapes.geometric}
\pagestyle{empty}

\begin{document}

\tikzset{
  rtria/.style={
    draw,
    shape border uses incircle,
    isosceles triangle,
    isosceles triangle apex angle=90,
    shape border rotate=-45,
    xshift=0.1cm,
    text width=1mm,
    inner sep=2pt
  }
}

\begin{tikzpicture}

\tikzstyle{every node}=[rtria,draw,minimum size=0.6cm];

% Draw the size 15 poplars
\node (n1) at (0,-0.4) {};
\node (n2) at (1,-0.4) {};
\node (n4) at (3,-0.4) {};
\node (n5) at (4,-0.4) {};
\node (n8) at (7,-0.4) {};
\node (n9) at (8,-0.4) {};
\node (n11) at (10,-0.4) {};

\tikzstyle{every node}=[draw,shape=circle,minimum size=0.75cm];

% Draw the other poplar heap nodes
\node (n3) at (2,1) {};
\node (n6) at (5,1) {};
\node (n7) at (6,2) {};
\node (n10) at (9,1) {};

% Links between the nodes
\draw (n7) -- (n6)
(n7) -- (n3)
(n6) -- (n5)
(n6) -- (n4)
(n3) -- (n2)
(n3) -- (n1)
(n10) -- (n9)
(n10) -- (n8);

% Draw the array representation of the poplar heap 
\tikzstyle{every node}=[draw,shape=rectangle,minimum width=1cm,minimum height=0.6cm,anchor=center];
\foreach \x [count=\idx] in {0,1,,0,2,,,0,1,,0}
  \node (r\idx) at (\idx-1,-1.2){\x};

\end{tikzpicture}
\end{document}
